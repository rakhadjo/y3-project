\chapter{Introduction}
\label{cha:intro}
\section{Motivation}
To simulate is the act of imitating real world processes or systems over time \cite{banks1996discrete}. It is also well understood that simulations are important in a myriad of contexts. In Science, Technology and Engineering subjects, simulations may be used for performance optimization or tuning, \cite{srinivasan2021words} or also ensuring systems provide acceptable levels of safetiness through the discipline of safety engineering \cite{verma2010reliability}. Simulations may also appear in scientific modelling of natural systems and economics \cite{orcutt1960simulation}. In general, however, simulations are used whenever the real system in question cannot be interfaced with (for whatever reason) \cite{srinivasan2021words}. 
\\ \\
One type of simulating is through the use of computer simulations. Simulating using computers started garnering popularity as far as World War II, when Jon Von Neumann and Stanislaw Ulam were investigating the complex behavior of neutrons during the Manhattan Project \cite{robeson}. Since then, simulations and modelling have undoubtedly become paramount in a variety of fields today, especially in dealing with complex systems.
\\ \\
Complex systems revolve around us - and oftentimes they can be generated by simple mechanisms \cite{southwell}. There are a handful of ways where such systems can be modelled using simple rules, but one such way of modelling complex systems can be done by cellular automata.
\\ \\
Commonly used as computer simulations, cellular automata (CA) is a discrete computation model studied in Automata Theory \cite{wolfram}. For  most simulations, CA would often appear as a grid of cells, with each cell in having one state from a finite set of states (e.g. \textit{on} state, or \textit{off} state). The states of these cells in the aforementioned grid may change each time-step (or generation), according to a set of transition rules and its neighbouring cells. 
\\ \\
CAs have demonstrated their simulation prowess in a variety of fields. They've oftentimes been applied to simulate systems in physics and biology \cite{wolfram}, forest fire propagation \cite{freire}, population dynamics \cite{mavroudi} \cite{moreno}, and can even be implemented in game development \cite{kowalski} among many others. Therefore, on the basis of the variety of a cellular automata's uses, this project's aim is to build a universal cellular automata simulator, where the user can define their own cellular automata rules in how the simulator would step. 
\section{Objectives}
\subsection{Objective 1: Research \& Development}
To achieve the aim of building a universal cellular automata simulator, background research is essential in understanding the underlying concepts of a cellular automata, its different types, and identify its various applications in addition to understanding what technologies are there to help me. The simulator should also allow the users to input their own rules in the simulation.
\\ \\
Therefore, an outline of my objectives in the scope of Research would be to:
\begin{itemize}
    \item gain a deeper understanding of cellular automata - understand what they are and different types of them, how they operate, their various applications, and how they are typically implemented in code
    \item hone and develop skills in JavaScript (JS) and HTML,
    \item browse various existing frameworks that might work as building blocks for the application
    \item come up with a rule format that is readable for humans
\end{itemize}
After research is done, then the goal will be to build a universal cellular automata simulator.
\subsection{Objective 2: Evaluation}
The final objective of the project is to evaluate the usefulness of this tool to the computer science university students and gather feedback on it. I define usefulness in this case as how easy the users can understand how to use the simulator. In particular, I'd like to obtain feedback on how easy to understand and use the rules are. Qualitative information will be gathered through small-scale interviews. 
\\ \\
The population of computer science students is chosen as the target audience. This is because the end product involves interfacing with text written in the JSON format, and of course cellular automata. The aforementioned group is therefore more likely to understand the JSON format and cellular automata.
\\ \\
Qualitative evaluation will be performed by short interviews where the user will use the system. Without the appropriate user feedback, the system's usability will forever remain subjective speculation to me, the developer. 